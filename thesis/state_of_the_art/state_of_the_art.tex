\documentclass{article}

\begin{document}
	\section{Background}

	Art/Background + Related Work + Motivație + Cazuri de utilizare
	\section{State of the art}

	\subsection{Context}
	\subsection{Yocto}
        \subsubsection{Build directory}
            After running the configuration scripts a build folder is created containing among others configuration files. When bitbake completes the tasks, the build directory is populated with the files downloaded by each recipe, a cache and a tmp folder.

	\subsection{Poky}
		\subsubsection{BitBake}
		\subsubsection{OpenEmbedded-Core}
		\subsubsection{Metadata}
			-> TODO: parallel between file systems and abstractions <<<<<<<<<<<<< Astazi

	\subsection{Specifications}
		\subsubsection{CGL - Carrier Grade Linux}
			CGL is defined sets of requirements and gaps. Each set represents a separate category or objective that aims to provide a different functionality vital to the system and can be made of one or more requirements and gaps. The objectives are:

			1.Availability
			2.Clustering
			3.Serviceability
			4.Performance
			5.Hardware support
			6.Security
			7.Standard implementation

			A requirement represents an aspect, application or feature essential for providing one of the above functionality that has at least an active or open source implementation, while a gap does not currently have an active or open source implementation.
			A requirement contains:
			ID: unique identifier composed of some identification elements
			Name: short description
			Category: one of the objectives
			Priority: marks if a requirement has to be implemented or not
			Description: detailed description

			Gaps have a similar format.

		\subsubsection{AGL - Automotive Grade Linux}
			Automotive Grade Linux (AGL) is a Linux Foundation workgroup dedicated to creating open source software solutions for automotive applications. The current target for AGL is In-Vehicle-Infotainment (IVI) systems, additional use cases such as instrument clusters and telematics systems will eventually be supported.

			Structured as several layers that focus on a specific part of the system. AGL layers are:

			1.App/HMI
			2.Application framework
			3.Services
			4.Operating system

		Each layer consists of several components and their description, each component might have one or more levels of sub-components.
		TODO: tree representation of AGL
		The conceptual difference between AGL and CGL specifications is that AGL looks in detail at a monolithic system and its main software components - the particular details of a certain kind of system. CGL however has a broader approach referring to its main building blocks as objectives - a dynamic, feature oriented classification of requirements. AGL defines characteristics and parameters for building blocks or modules, CGL defines a set of actions and how they should behave so that systems main objectives are accomplished.

		\subsubsection{GenIVI}

	\subsection{Testing platforms}
		\subsubsection{LTP}
		\subsubsection{Fuego}
			\paragraph{Jenkins}
				\emph{Jenkins} is an open source automation server written in Java. Jenkins is used for automating tasks needed for continuous integration and continuous delivery. It offers a simple to use plugin manager that allows plugin written in several programming languages.
		\subsubsection{LAVA}

\end{document}

https://www.genivi.org/faq
https://polysync.io/
https://www.cip-project.org/
http://events.linuxfoundation.org/sites/events/files/slides/Introducing%20the%20Civil%20Infrastructure%20Platform.pdf
https://kernelci.org/

\documentclass{article}

\begin{document}

	\section{Implementation}

		Ellida is a framework that aims to bridge formal specification files with standalone tests or already existing testing suites. The main function it fulfills is to abstract away the human readable format of the specification and convert it in a parsable mapping between requirements and concrete tests.


		\subsection{Architecture}
		+ Specification - structured document that contains a series of requirements that can depend on one another
		+ Test provider - any source of tests such as testing suites, other frameworks, databases or simply folders of standalone testing files can be registered as providers
		+ Abstraction - mapping between test providers, individual tests and requirements.

		\subsection{Framework}

			\subsubsection{Specification parser}
				The specifications come in various formats. The framework uses an automated parser to process the contents of a specification and facilitates abstraction updating when the specification changes.
			\subsubsection{Database}
				The tests are structured as a directory tree with JSON files representing parts of the specification. Each specification stands as a root directory with the rest of the tree made so that it maps the requirements.
				Screenshots of directory trees

			\subsubsection{Manager}

				Role: manage the test database by adding/removing/changing tests
				The test manager is the only component with which the user needs to interact (through a GUI), it provides ease of access to the database and formats any new tests so that the framework can use them.

				How and where should the mapping between a test file and a .json file be made?
				1.Inside the .json specify the path or name of the test
				2.Let the manager handle the mapping at each run using the .json ID - how does it know what tests to use?

				[?] Including python modules from other folders seems to be discouraged in Yocto, classes and recipes fully implement the components they need. Is it because so it stays more modular or does it have something to do with the way python imports modules (using sys and append the path to the python path at runtime seems a bad practice)?

			\subsubsection{Test suite}


			\subsubsection{User interface}
				The UI comes in the form of a single-page application (SPA) which is a web application that fits on a single web page with the goal of providing a user experience similar to that of a desktop application. The code is written using AngularJS library.

		\subsection{Integration and API}

\end{document}